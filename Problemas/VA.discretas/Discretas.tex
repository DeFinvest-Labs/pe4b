% Options for packages loaded elsewhere
\PassOptionsToPackage{unicode}{hyperref}
\PassOptionsToPackage{hyphens}{url}
%
\documentclass[
]{article}
\usepackage{lmodern}
\usepackage{amssymb,amsmath}
\usepackage{ifxetex,ifluatex}
\ifnum 0\ifxetex 1\fi\ifluatex 1\fi=0 % if pdftex
  \usepackage[T1]{fontenc}
  \usepackage[utf8]{inputenc}
  \usepackage{textcomp} % provide euro and other symbols
\else % if luatex or xetex
  \usepackage{unicode-math}
  \defaultfontfeatures{Scale=MatchLowercase}
  \defaultfontfeatures[\rmfamily]{Ligatures=TeX,Scale=1}
\fi
% Use upquote if available, for straight quotes in verbatim environments
\IfFileExists{upquote.sty}{\usepackage{upquote}}{}
\IfFileExists{microtype.sty}{% use microtype if available
  \usepackage[]{microtype}
  \UseMicrotypeSet[protrusion]{basicmath} % disable protrusion for tt fonts
}{}
\makeatletter
\@ifundefined{KOMAClassName}{% if non-KOMA class
  \IfFileExists{parskip.sty}{%
    \usepackage{parskip}
  }{% else
    \setlength{\parindent}{0pt}
    \setlength{\parskip}{6pt plus 2pt minus 1pt}}
}{% if KOMA class
  \KOMAoptions{parskip=half}}
\makeatother
\usepackage{xcolor}
\IfFileExists{xurl.sty}{\usepackage{xurl}}{} % add URL line breaks if available
\IfFileExists{bookmark.sty}{\usepackage{bookmark}}{\usepackage{hyperref}}
\hypersetup{
  pdftitle={Problemas variables aleatorias discretas},
  hidelinks,
  pdfcreator={LaTeX via pandoc}}
\urlstyle{same} % disable monospaced font for URLs
\usepackage[margin=1in]{geometry}
\usepackage{graphicx}
\makeatletter
\def\maxwidth{\ifdim\Gin@nat@width>\linewidth\linewidth\else\Gin@nat@width\fi}
\def\maxheight{\ifdim\Gin@nat@height>\textheight\textheight\else\Gin@nat@height\fi}
\makeatother
% Scale images if necessary, so that they will not overflow the page
% margins by default, and it is still possible to overwrite the defaults
% using explicit options in \includegraphics[width, height, ...]{}
\setkeys{Gin}{width=\maxwidth,height=\maxheight,keepaspectratio}
% Set default figure placement to htbp
\makeatletter
\def\fps@figure{htbp}
\makeatother
\setlength{\emergencystretch}{3em} % prevent overfull lines
\providecommand{\tightlist}{%
  \setlength{\itemsep}{0pt}\setlength{\parskip}{0pt}}
\setcounter{secnumdepth}{-\maxdimen} % remove section numbering

\title{Problemas variables aleatorias discretas}
\author{}
\date{\vspace{-2.5em}}

\begin{document}
\maketitle

\begin{enumerate}
\def\labelenumi{\arabic{enumi}.}
\item
  Un fabricante de patinetes eléctricos calcula la proporción del número
  de patinetes vendidos (\(X\)) que han reclamado en el periodo de
  garantía a corregir alguna avería incluida en la garantía. Los datos
  se resumen en la siguiente tabla:

  \begin{center}
  \begin{tabular}{|l||r|r|r|r|r|}
  \multicolumn{6}{c}{\vphantom{R}}\\
  \hline
  Número de veces $x$    & 0    & 1    &  2   &  3   &  4 \\
  \hline
  Proporción    $P(X=x)$ & 0.38 & 0.37 & 0.19 & 0.05 & 0.01
  \\\hline \multicolumn{6}{c}{\vphantom{R}}
  \end{tabular}
  \end{center}

  \begin{enumerate}
  \def\labelenumii{\alph{enumii}.}
  \tightlist
  \item
    Dibuja el gráfico de la función de distribución de probabilidad.
  \item
    Dibuja el gráfico de la función de distribución de probabilidad
    acumulada.
  \item
    Calcula media de el número de reclamaciones en el periodo de
    garantía.
  \item
    Calcula varianza de el número de reclamaciones en le periodo de
    garantía.
  \end{enumerate}
\item
  La empresa CAT4ALL de ágapes/catering para eventos: fiestas, bodas,
  congresos. Contrata a la empresa CATERISIMO para que le le consiga
  clientes por su portal de internet. El número de contratos mensuales
  que han conseguido a través de la mediación de de CATERISIMO han sido:

  \begin{center}
  \begin{tabular}{|l||r|r|r|r|r|r|r|}
  \multicolumn{8}{c}{\vphantom{R}}\\
  \hline
  Número de eventos $x$ & 14  & 15  &  16 &  17 &  18 & 19 & 20 \\
  \hline
  Proporción  $P(X=x)$      & 0.06&0.13 &0.24 &0.24 &0.12 &0.1 &0.11
  \\\hline \multicolumn{8}{c}{\vphantom{R}}
  \end{tabular}
  \end{center}

  \begin{enumerate}
  \def\labelenumii{\alph{enumii}.}
  \tightlist
  \item
    Dibuja el gráfico de la función de distribución de probabilidad.
  \item
    Dibuja el gráfico de la función de distribución de probabilidad
    acumulada.
  \item
    Calcula media del número de contratos por mes.
  \item
    Calcula varianza del número de contratos por mes.
  \item
    Reproduce con un ordenador (R, python, Excel, Libre Office, Google
    Spreadsheets,\ldots), los cálculos de la media y la varianza,
  \end{enumerate}
\item
  Ha llegado al puerto de Barcelona una partida muy grande de latas de
  caviar CAVIARFRIO de 1KG. El transportista sabe que la partida
  contiene un 1\% de latas con defectos visibles en el exterior (golpes,
  deformaciones, arañazos). Supongamos que encargamos \(n=3\) latas que
  se escogen al azar de esta partida. Sea \(X\) el número de latas
  defectuosas entre las \(n\) latas.

  \begin{enumerate}
  \def\labelenumii{\alph{enumii}.}
  \tightlist
  \item
    Modelar mediante una distribución binomial la función de
    probabilidad del número de latas con defecto entre las tres.
  \item
    Calcule la función probabilidad del número de latas defectuosas.
  \item
    Calcule la función de distribución del número de latas defectuosas.
  \item
    Calcule la media del número de latas defectuosas.
  \item
    Calcule la varianza y la desviación típica del número de latas
    defectuosas.
  \end{enumerate}
\end{enumerate}

\newpage

\begin{enumerate}
\def\labelenumi{\arabic{enumi}.}
\setcounter{enumi}{3}
\tightlist
\item
  Un asesor de un Banco recibe el encargado de llamar a sus clientes
  para ofrecerles un crédito los días previos al Black Friday.
  Supongamos que tiene una gran lista de clientes y que los va llamando
  de forma sucesiva hasta que consigue contratar el primer crédito y que
  la probabilidad de que un cliente escogido al azar contrate el
  producto es \(p=0.15\). Sea \(X\) el número de llamadas fracasadas
  para conseguir la primera venta.

  \begin{enumerate}
  \def\labelenumii{\alph{enumii}.}
  \tightlist
  \item
    Modelar con alguna distribución notable discreta la distribución de
    probabilidad de \(X\). Si es necesario añadir alguna condición
    adicional al problema.
  \item
    Calcular la función de probabilidad y de probabilidad acumulada de
    \(X\).
  \item
    Calcular la esperanza y la desviación típica de \(X\).
  \item
    El asesor ya ha hecho 20 llamadas consecutivas sin éxito ¿Cuál es la
    probabilidad de que eso pase?
  \end{enumerate}
\item
  Nuestro socio Pablo Andrés Obrador está considerando invertir 1000 € y
  está escogiendo entre tres fondos de inversión que en su publicidad
  ofrecen como ejemplo estás tres modalidades: • Fondo 1: Un beneficio
  de 10000€ con probabilidad 20\% o en caso contrario perder todas la
  inversión. • Fondo 2: Un beneficio de 10000€ con probabilidad 45\%, o
  un beneficio de 500€ con probabilidad 30\%, o perder 500€ con
  probabilidad del 25\%. • Fondo 3: Un beneficio seguro de 400€.

  \begin{enumerate}
  \def\labelenumii{\alph{enumii}.}
  \tightlist
  \item
    Calcula los valores esperados y desviaciones típica de los
    beneficios en cada caso.
  \item
    ¿Qué elección le recomendarías a Pablo Andrés?
  \end{enumerate}
\item
  Un comercio de venta de Tabacos decide también hacer de puesto de
  recogida de pequeñas entregas de una distribuidora. El número de
  paquetes diario que le llegan para que los destinatarios los recojan
  sigue es una variable aleatoria \(X\) con \(\lambda=13\).

  \begin{enumerate}
  \def\labelenumii{\alph{enumii}.}
  \tightlist
  \item
    Modela función de probabilidad de \(X\) con una distribución
    Poisson.
  \item
    Calcular \(E(X)\) y \(Var(X)\).
  \item
    El comerciante dispone de un espacio para almacenar unos 20
    paquetes. ¿Cuál es la probabilidad de que lleguen más paquetes y
    ocupe todo el espacio? Utiliza un ordenador para este cálculo
  \end{enumerate}
\end{enumerate}

\end{document}
